 \documentclass{beamer}
\usefonttheme{professionalfonts}
\usepackage{agda}
\usepackage{catchfilebetweentags}
\usetheme{CambridgeUS}
\usecolortheme{seahorse}

\usepackage{amssymb}
\usepackage{bbm}
\usepackage[greek,english]{babel}

\beamertemplatenavigationsymbolsempty

\title{Dependent Types in GHC}
\author{John Leo}
\institute{Halfaya Research}
\date{April 8, 2017}
 
\begin{document}
 
\frame{\titlepage}
 
\begin{frame}\frametitle{References}
This talk:
\begin{itemize}
\item https://github.com/halfaya/BayHac
\end{itemize}
References:
\begin{itemize}
\item https://github.com/halfaya/BayHac/blob/master/references.md
\end{itemize}
\end{frame}

\begin{frame}\frametitle{Outline}
\begin{itemize}
\item Vectors in Agda
\item Big Picture
\item Dependent Types in Haskell
\item Vectors in today's Haskell
\item Vectors in Dependent Haskell
\end{itemize}
\end{frame}

\begin{frame}\frametitle{What are Dependent Types?}
\ExecuteMetaData[../agda/latex/BayHac.tex]{nat}
\ExecuteMetaData[../agda/latex/BayHac.tex]{vec}
\end{frame}

\begin{frame}\frametitle{Vector Append}
\ExecuteMetaData[../agda/latex/BayHac.tex]{plus}
\ExecuteMetaData[../agda/latex/BayHac.tex]{append}
\end{frame}

\begin{frame}\frametitle{Vector Lookup}
\ExecuteMetaData[../agda/latex/BayHac.tex]{fin}
\ExecuteMetaData[../agda/latex/BayHac.tex]{lookup}
\end{frame}

\begin{frame}\frametitle{Some Basic Types}
\ExecuteMetaData[../agda/latex/BayHac.tex]{bool}
\ExecuteMetaData[../agda/latex/BayHac.tex]{eq}
\end{frame}

\begin{frame}\frametitle{Vector Lookup 2}
\ExecuteMetaData[../agda/latex/BayHac.tex]{lt}
\ExecuteMetaData[../agda/latex/BayHac.tex]{lookup2}
\end{frame}

\begin{frame}\frametitle{Why Use Dependent Types?}
\begin{itemize}
\item More expressive and precise
\item Propositions as Types (PAT, Curry-Howard Correspondence):
\begin{itemize}
\item Universal Quantification ($\forall$) corresponds to $\Pi$ types.\\
  $\Pi_{x : A} B(x)$ or $(x : A) \to B_x$.
\item Existential Quantification ($\exists$) corresponds to $\Sigma$ types.\\
  $\Sigma_{x : A} B(x)$ or $(x : A) \times B_x$.
\end{itemize}
\end{itemize}
\end{frame}

\begin{frame}\frametitle{Dependent Types are Not New}
\begin{columns}
\column{0.5\textwidth}
\begin{description}[align=right]
\item[1971] \textcolor{blue}{System F}
\item[1971] \textcolor{blue}{Martin-L\"of Type Theory}
\item[1972] \textcolor{brown}{LCF/ML}
\item[1978] \textcolor{blue}{Hindley-Milner (H 1969)}
\item[1979] \textcolor{blue}{Constructive math and computer programming}
\item[1982] \textcolor{blue}{Damas-Milner}
\item[1983] \textcolor{brown}{Standard ML}
\item[1984] \textcolor{blue}{Calculus of Constructions}
\item[1984] \textcolor{red}{NuPrl}
\item[1985] \textcolor{brown}{Miranda}
\item[1987] \textcolor{brown}{Caml}
\item[1988] \textcolor{blue}{CiC}
\end{description}
 
\column{0.5\textwidth}
\begin{description}[align=right]
\item[1989] \textcolor{red}{Coq}
\item[1990] \textcolor{brown}{Haskell}
\item[1990] \textcolor{blue}{Nordstr\"om, et al}, \textcolor{red}{ALF}
\item[1998] \textcolor{red}{Cayenne}
\item[1999] \textcolor{red}{Agda 1}
\item[1991] \textcolor{brown}{Caml Light}
\item[1996] \textcolor{brown}{OCaml}
\item[2007] \textcolor{red}{Agda 2}
\item[2011] \textcolor{red}{Idris}
\item[2013] \textcolor{blue}{Homotopy Type Theory}
\item[2013] \textcolor{red}{Lean}
\item[2015] \textcolor{blue}{Cubical Type Theory}
\end{description}
\end{columns}
\end{frame}

\begin{frame}\frametitle{The Golden Age is Now}
\begin{itemize}
\item Increased use of FP in industry.\\
  Big Data, Finance, Security
\item Better correctness guarantees for software.\\
  CompCert, DeepSpec, etc.
\item Mechanical verification of mathematics.\\
  Four Color Theorem, Feit-Thompson, BigProof
\item Natural Language Processing.\\
  Grammatical Framework
\item Theoretical work.\\
  HoTT, Cubical Type Theory, Category Theory and FP, etc.
\end{itemize}
\end{frame}

\begin{frame}\frametitle{Robert Harper}
\begin{quote}
Eventually all the arbitrary programming languages are going to be just swept away with the oceans,
and we will have the permanence of constructive, intuistionistic type theory as the master theory
of computation---without doubt, in my mind, no question.  So, from my point of view---this is a personal
statement---working in anything else is a waste of time.
\end{quote}

CMU Homotopy Type Theory lecture 1, 52:56--53:20.
\end{frame}

\begin{frame}\frametitle{Dependent Types in Haskell}
Richard Eisenberg's PhD Thesis  
\begin{enumerate}
\item Introduction
\item Preliminaries
\item Motivation
\item Dependent Haskell
\item PICO: The Intermediate Language
\item Type Inference and Elaboration, or how to BAKE a PICO
\item Implementation
\item Related and Future Work
\end{enumerate}
\end{frame}

\begin{frame}\frametitle{Time Line}
From Richard Eisenberg's Blog:

\bigskip

\textbf{When can we expect dependent types in GHC?}

\bigskip

The short answer:\\
GHC 8.4 (2018) at the very earliest.\\
More likely 8.6 or 8.8 (2019-20).
\end{frame}

\begin{frame}[fragile]\frametitle{Nat}
\begin{semiverbatim}
\{-# LANGUAGE UnicodeSyntax, ExplicitForAll, GADTs,
              TypeFamilies, TypeOperators, TypeInType #-\}
import Data.Kind (Type)

data Nat ∷ Type where
  Zero ∷ Nat
  Succ ∷ Nat → Nat
\end{semiverbatim}
\ExecuteMetaData[../agda/latex/BayHac.tex]{nat}
\end{frame}

\begin{frame}[fragile]\frametitle{Vector}
\begin{semiverbatim}
data Vec ∷ Type → Nat → Type where
  Nil  ∷ ∀ (a ∷ Type). Vec a 'Zero
  (:>) ∷ ∀ (a ∷ Type)(n ∷ Nat).
    a → Vec a n → Vec a ('Succ n)
\end{semiverbatim}
\ExecuteMetaData[../agda/latex/BayHac.tex]{vec}
\end{frame}

\begin{frame}[fragile]\frametitle{Plus}
\begin{semiverbatim}
type family (m ∷ Nat) + (n ∷ Nat) ∷ Nat where
  'Zero   + n = n
  'Succ m + n = 'Succ (m + n)
\end{semiverbatim}
\ExecuteMetaData[../agda/latex/BayHac.tex]{plus}
\end{frame}

\begin{frame}[fragile]\frametitle{Append}
\begin{semiverbatim}
(++) ∷ ∀ (a ∷ Type)(m ∷ Nat)(n ∷ Nat).
       Vec a m → Vec a n → Vec a (m + n)
Nil    ++ v = v
x :> u ++ v = x :> (u ++ v)
\end{semiverbatim}
\ExecuteMetaData[../agda/latex/BayHac.tex]{append}
\end{frame}

\begin{frame}[fragile]\frametitle{Fin}
\begin{semiverbatim}
data Fin ∷ Nat → Type where
  FZero ∷ ∀ (n ∷ Nat).         Fin (Succ n)
  FSucc ∷ ∀ (n ∷ Nat). Fin n → Fin (Succ n)
\end{semiverbatim}
\ExecuteMetaData[../agda/latex/BayHac.tex]{fin}
\end{frame}

\begin{frame}[fragile]\frametitle{Lookup}
\begin{semiverbatim}
lookup ∷ ∀ (a ∷ Type)(n ∷ Nat). Fin n → Vec a n → a
lookup FZero     (x :> _)  = x
lookup (FSucc n) (_ :> xs) = lookup n xs
\end{semiverbatim}
\ExecuteMetaData[../agda/latex/BayHac.tex]{lookup}
\end{frame}

\begin{frame}[fragile]\frametitle{$\equiv$ and $<$}
\begin{semiverbatim}
data (a ∷ k) ≡ (b ∷ k) where
  Refl ∷ ∀ (k ∷ Type)(a ∷ k). a ≡ a

type family (m ∷ Nat) < (n ∷ Nat) ∷ Bool where
  _         < 'Zero     = False
  'Zero     < ('Succ _) = True
  ('Succ m) < ('Succ n) = m < n
\end{semiverbatim}
\ExecuteMetaData[../agda/latex/BayHac.tex]{eq}
\ExecuteMetaData[../agda/latex/BayHac.tex]{lt}
\end{frame}

\begin{frame}[fragile]\frametitle{Attempt at Lookup'}
\begin{semiverbatim}
lookupBad ∷ ∀ (a ∷ Type)(m ∷ Nat)(n ∷ Nat).
            m → m < n ≡ True → Vec a n → a

  • Expected a type, but ‘m’ has kind ‘Nat’
  • In the type signature:
      lookupBad ∷ ∀ (a ∷ Type) (m ∷ Nat) (n ∷ Nat).
                   m → (m < n) ≡ True → Vec a n → a
> :k (→)
(→) ∷ Type → Type → Type
\end{semiverbatim}
\ExecuteMetaData[../agda/latex/BayHac.tex]{lookup2}
\end{frame}

\begin{frame}[fragile]\frametitle{Another Attempt}
\begin{semiverbatim}
  type family LookupUp (a ∷ Type) (m ∷ Nat) (n ∷ Nat)
                       (p ∷ m < n ≡ 'True) 
                       (v ∷ Vec a n) :: a where

  LookupUp a 'Zero     ('Succ n) 'Refl (x :> _)  = x
  LookupUp a ('Succ m) ('Succ n) p     (_ :> xs) =
    LookupUp a m n p xs
\end{semiverbatim}
\end{frame}

\begin{frame}[fragile]\frametitle{Singleton Nat}
\begin{semiverbatim}
data SNat ∷ Nat → Type where
  SZero ∷ SNat 'Zero
  SSucc ∷ ∀ (n ∷ Nat). SNat n → SNat ('Succ n)
\end{semiverbatim}
\end{frame}

\begin{frame}[fragile]\frametitle{Lookup'}
\begin{semiverbatim}
lookup' ∷ ∀ (a ∷ Type)(m ∷ Nat)(n ∷ Nat).
           SNat m → m < n ≡ True → Vec a n → a
lookup' SZero     Refl (x :> _)  = x
lookup' (SSucc m) Refl (_ :> xs) = lookup' m Refl xs
\end{semiverbatim}
\ExecuteMetaData[../agda/latex/BayHac.tex]{lookup2}
\end{frame}

\begin{frame}[fragile]\frametitle{Another Variation}
\begin{semiverbatim}
nth ∷ ∀ (a ∷ Type)(m ∷ Nat)(n ∷ Nat).
      (m < n) ~ 'True => SNat m → Vec a n → a
nth SZero     (a :> _)  = a
nth (SSucc m) (_ :> as) = nth m as
\end{semiverbatim}
\end{frame}

\begin{frame}[fragile]\frametitle{Lookup' vs Nth}
\begin{semiverbatim}
lookup' ∷ ∀ (a ∷ Type)(m ∷ Nat)(n ∷ Nat).
           SNat m → m < n ≡ True → Vec a n → a
lookup' _         _    Nil       = undefined
lookup' SZero     Refl (x :> _)  = x
lookup' (SSucc m) Refl (_ :> xs) = lookup' m Refl xs

nth ∷ ∀ (a ∷ Type)(m ∷ Nat)(n ∷ Nat).
      (m < n) ~ 'True => SNat m → Vec a n → a
nth _         Nil       =  undefined
nth SZero     (a :> _)  = a
nth (SSucc m) (_ :> as) = nth m as
• Couldn't match type ‘'True’ with ‘'False’
  Inaccessible code in
    a pattern with constructor: Nil ∷ ∀ a. Vec a 'Zero,
    in an equation for ‘nth’
\end{semiverbatim}
\end{frame}

\begin{frame}[fragile]\frametitle{Vectors in Dependent Haskell}
\begin{semiverbatim}
type family (m ∷ Nat) + (n ∷ Nat) ∷ Nat where
  'Zero   + n = n
  'Succ m + n = 'Succ (m + n)

(+) ∷ Nat → Nat → Nat
Zero   + m = m
Succ n + m = Succ (n + m)
\end{semiverbatim}
\ExecuteMetaData[../agda/latex/BayHac.tex]{plus}
\end{frame}

\begin{frame}[fragile]\frametitle{Append}
\begin{semiverbatim}
(++) ∷ ∀ (a ∷ Type)(m ∷ Nat)(n ∷ Nat).
       Vec a m → Vec a n → Vec a (m + n)
Nil    ++ v = v
x :> u ++ v = x :> (u ++ v)

(++) ∷ ∀ (a ∷ Type)(m ∷ Nat)(n ∷ Nat).
       Vec a m → Vec a n → Vec a (m '+ n)
Nil    ++ v = v
x :> u ++ v = x :> (u ++ v)
\end{semiverbatim}
\ExecuteMetaData[../agda/latex/BayHac.tex]{append}
\end{frame}

\begin{frame}[fragile]\frametitle{Lookup'}
\begin{semiverbatim}
lookup' ∷ ∀ (a ∷ Type)(m ∷ Nat)(n ∷ Nat).
          SNat m → m < n ≡ True → Vec a n → a
lookup' SZero     Refl (x :> _)  = x
lookup' (SSucc m) Refl (_ :> xs) = lookup' m Refl xs

lookup' ∷ ∀ (a ∷ Type)(n ∷ Nat). Π (m ∷ Nat) →
          m < n ≡ True → Vec a n → a
lookup' Zero     Refl (x :> _)  = x
lookup' (Succ m) Refl (_ :> xs) = lookup' m Refl xs
\end{semiverbatim}
\ExecuteMetaData[../agda/latex/BayHac.tex]{lookup2}
\end{frame}

\begin{frame}\frametitle{Conclusion}
This is just a tiny taste.
\bigskip  

See Richard's thesis and the other references for much more.  
\bigskip  

Try playing with a proof assistant such as Coq or Agda.\\
Software Foundations is a great place to start.
\end{frame}

\end{document}
